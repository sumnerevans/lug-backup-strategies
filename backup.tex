\documentclass{lug}

\usepackage{fontawesome}
\usepackage{etoolbox}
\usepackage{textcomp}
\usepackage[nodisplayskipstretch]{setspace}
\usepackage{xspace}
\usepackage{verbatim}
\usepackage{multicol}
\usepackage{soul}
\usepackage{attrib}

\usepackage{amsmath,amssymb,amsthm}

\usepackage[linesnumbered,commentsnumbered,ruled,vlined]{algorithm2e}
\newcommand\mycommfont[1]{\footnotesize\ttfamily\textcolor{blue}{#1}}
\SetCommentSty{mycommfont}
\SetKwComment{tcc}{ \# }{}
\SetKwComment{tcp}{ \# }{}

\usepackage{siunitx}

\usepackage{tikz}
\usepackage{pgfplots}
\usetikzlibrary{decorations.pathreplacing,calc,arrows.meta,shapes,graphs}

\AtBeginEnvironment{minted}{\singlespacing\fontsize{10}{10}\selectfont}

\makeatletter
\patchcmd{\beamer@sectionintoc}{\vskip1.5em}{\vskip0.5em}{}{}
\makeatother

% Math stuffs
\newcommand{\Z}{\mathbb{Z}}
\newcommand{\R}{\mathbb{R}}
\newcommand{\N}{\mathbb{N}}
\newcommand{\lcm}{\text{lcm}}
\newcommand{\Inn}{\text{Inn}}
\newcommand{\Aut}{\text{Aut}}
\newcommand{\Ker}{\text{Ker}\ }
\newcommand{\la}{\langle}
\newcommand{\ra}{\rangle}

\newcommand{\yournewcommand}[2]{Something #1, and #2}

\newenvironment{question}[1]{\par\textbf{Question #1.}\par}{}

\newcommand{\pmidg}[1]{\parbox{\widthof{#1}}{#1}}
\newcommand{\splitslide}[4]{
    \noindent
    \begin{minipage}{#1 \textwidth - #2 }
        #3
    \end{minipage}%
    \hspace{ \dimexpr #2 * 2 \relax }%
    \begin{minipage}{\textwidth - #1 \textwidth - #2 }
        #4
    \end{minipage}
}

\newcommand{\frameoutput}[1]{\frame{\colorbox{white}{#1}}}

\newcommand{\tikzmark}[1]{%
\tikz[baseline=-0.55ex,overlay,remember picture] \node[inner sep=0pt,] (#1)
{\vphantom{T}};
}

\newcommand{\braced}[3]{%
    \begin{tikzpicture}[overlay,remember picture]
        \draw [thick,decorate,decoration={brace,raise=1ex,amplitude=4pt},blue] (#2.south west-|T1.south west) -- node[anchor=west,left,xshift=-1.8ex,text=olive]{#3} (#1.north west-|T1.south west);
    \end{tikzpicture}
}

\newcommand{\make}{GNU \texttt{make}\xspace}

\title{Backup Strategies}
\author{Sumner Evans}
\institute{Mines Linux Users Group}

\begin{document}

\section{Backup Principles}

\begin{frame}[fragile]{Why backup? I}

    \textbf{Computers were a mistake}. \pause But the bigger mistake was to give humans
    control over the computers.

    \pause
    Sometimes certain humans may write a program along the lines of:
    \begin{minted}{python}
        with open("~/awesome", "w+") as f:
            f.writeline("Awesome program\n")
    \end{minted}

    \pause which doesn't do what you expect because by default Python doesn't
    expand \texttt{\textasciitilde} by default meaning this creates a directory
    named \texttt{\textasciitilde} in your working directory.

    \pause Naturally, to delete this directory, you would run \texttt{rm -rf
    \textasciitilde}, right?

    \pause (I may or may not have first-hand experience with this situation.)

\end{frame}

\begin{frame}[fragile]{Why backup? II}

    \textbf{Ransomware} protection.

    \pause If a malicious actor manages to
    encrypt a bunch of the files on your filesystem and demands money to get the
    key, you can just restore to a previous backup with minimal loss of
    productivity.

\end{frame}

\begin{frame}{Why not backup?}

    If you don't have any data that is important to you, \pause and like setting
    up your computer over and over again, \pause then you don't need a backup.

    \pause I'd bet that you have \textit{something} that you want to backup.

\end{frame}

\begin{frame}{Don't backup everything!}

    Backups can get bloated if you include too many unimportant files!

    \begin{itemize}[<+->]
        \item A very small number of your dotfiles are actually useful to be
            backed up.
        \item Likewise only a few files in \texttt{/etc} actually matter.
        \item \texttt{/var} sometimes contains things that are worth backing up.
        \item On Windows, it's pointless to backup
            \texttt{C:\textbackslash\textbackslash Program Files}.
    \end{itemize}

\end{frame}

\begin{frame}{The 3-2-1 rule}

    The \textbf{3-2-1 backup rule} states that you should:

    Keep at least \textbf{three} copies of your data \pause on at least
    \textbf{two} different storage media \pause and store at least \textbf{one}
    of the copies off-site.

    \pause This may sound daunting, but keep in mind that \textbf{any backup is
    better than no backup!} You have to start somewhere.

\end{frame}

\begin{frame}{A few other principles}

    \textbf{The best backups are automatic}, because otherwise you'll always
    default to ``oh, I can do that later''.

    \textbf{Tailor your backups to the data your are backing up.} For example,
    don't just backup all of the files that your database uses, rather export
    your database periodically and backup that export.

\end{frame}

\end{document}
% Local Variables:
% TeX-command-extra-options: "-shell-escape"
% End:
