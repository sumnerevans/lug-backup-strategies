\documentclass{lug}

\usepackage{fontawesome}
\usepackage{etoolbox}
\usepackage{textcomp}
\usepackage[nodisplayskipstretch]{setspace}
\usepackage{xspace}
\usepackage{verbatim}
\usepackage{multicol}
\usepackage{soul}
\usepackage{attrib}

\usepackage{amsmath,amssymb,amsthm}

\usepackage[linesnumbered,commentsnumbered,ruled,vlined]{algorithm2e}
\newcommand\mycommfont[1]{\footnotesize\ttfamily\textcolor{blue}{#1}}
\SetCommentSty{mycommfont}
\SetKwComment{tcc}{ \# }{}
\SetKwComment{tcp}{ \# }{}

\usepackage{siunitx}

\usepackage{tikz}
\usepackage{pgfplots}
\usetikzlibrary{decorations.pathreplacing,calc,arrows.meta,shapes,graphs}

\AtBeginEnvironment{minted}{\singlespacing\fontsize{10}{10}\selectfont}

\makeatletter
\patchcmd{\beamer@sectionintoc}{\vskip1.5em}{\vskip0.5em}{}{}
\makeatother

% Math stuffs
\newcommand{\Z}{\mathbb{Z}}
\newcommand{\R}{\mathbb{R}}
\newcommand{\N}{\mathbb{N}}
\newcommand{\lcm}{\text{lcm}}
\newcommand{\Inn}{\text{Inn}}
\newcommand{\Aut}{\text{Aut}}
\newcommand{\Ker}{\text{Ker}\ }
\newcommand{\la}{\langle}
\newcommand{\ra}{\rangle}

\newcommand{\yournewcommand}[2]{Something #1, and #2}

\newenvironment{question}[1]{\par\textbf{Question #1.}\par}{}

\newcommand{\pmidg}[1]{\parbox{\widthof{#1}}{#1}}
\newcommand{\splitslide}[4]{
    \noindent
    \begin{minipage}{#1 \textwidth - #2 }
        #3
    \end{minipage}%
    \hspace{ \dimexpr #2 * 2 \relax }%
    \begin{minipage}{\textwidth - #1 \textwidth - #2 }
        #4
    \end{minipage}
}

\newcommand{\frameoutput}[1]{\frame{\colorbox{white}{#1}}}

\newcommand{\tikzmark}[1]{%
\tikz[baseline=-0.55ex,overlay,remember picture] \node[inner sep=0pt,] (#1)
{\vphantom{T}};
}

\newcommand{\braced}[3]{%
    \begin{tikzpicture}[overlay,remember picture]
        \draw [thick,decorate,decoration={brace,raise=1ex,amplitude=4pt},blue] (#2.south west-|T1.south west) -- node[anchor=west,left,xshift=-1.8ex,text=olive]{#3} (#1.north west-|T1.south west);
    \end{tikzpicture}
}

\newcommand{\make}{GNU \texttt{make}\xspace}

\title{Backup Strategies}
\author{Sumner Evans}
\institute{Mines Linux Users Group}

\begin{document}

\section{Backup Principles}

\begin{frame}[fragile]{Why backup? I}

    \textbf{Computers were a mistake}. \pause But the bigger mistake was to give humans
    control over the computers.

    \pause
    Sometimes certain humans may write a program along the lines of:
    \begin{minted}{python}
        with open("~/awesome", "w+") as f:
            f.writeline("Awesome program\n")
    \end{minted}

    \pause which doesn't do what you expect because by default Python doesn't
    expand \texttt{\textasciitilde} by default meaning this creates a directory
    named \texttt{\textasciitilde} in your working directory.

    \pause Naturally, to delete this directory, you would run \texttt{rm -rf
    \textasciitilde}, right?

    \pause (I may or may not have first-hand experience with this situation.)

\end{frame}

\begin{frame}[fragile]{Why backup? II}

    \textbf{Ransomware} protection.

    \pause If a malicious actor manages to
    encrypt a bunch of the files on your filesystem and demands money to get the
    key, you can just restore to a previous backup with minimal loss of
    productivity.

\end{frame}

\begin{frame}{Why not backup?}

    If you don't have any data that is important to you, \pause and like setting
    up your computer over and over again, \pause then you don't need a backup.

    \pause I'd bet that you have \textit{something} that you want to backup.

\end{frame}

\begin{frame}{Don't backup everything!}

    Backups can get bloated if you include too many unimportant files!
    \pause

    \begin{itemize}[<+->]
        \item A very small number of your dotfiles are actually useful to be
            backed up.
        \item Likewise only a few files in \texttt{/etc} actually matter.
        \item \texttt{/var} sometimes contains things that are worth backing up.
        \item On Windows, it's pointless to backup
            \texttt{C:\textbackslash\textbackslash Program Files}.
    \end{itemize}

\end{frame}

\begin{frame}{The 3-2-1 rule}

    The \textbf{3-2-1 backup rule} states that you should: \pause

    Keep at least \textbf{three} copies of your data \pause on at least
    \textbf{two} different storage media \pause and store at least \textbf{one}
    of the copies off-site.

    \pause This may sound daunting, but keep in mind that \textbf{any backup is
    better than no backup!} You have to start somewhere.

\end{frame}

\begin{frame}{File sync is not backup}

    File sync is not a backup strategy! \pause It's an catastrophe distribution
    system.

\end{frame}

\begin{frame}{A few other principles}

    \textbf{The best backups are automatic}, because otherwise you'll always
    default to ``oh, I can do that later''.

    \pause
    \textbf{Tailor your backups to the data you are backing up.} For example,
    don't just backup all of the files that your database uses, rather export
    your database periodically and backup that export.

    \pause
    \textbf{Test your backups before you need them!} You want to be confident in
    your backups. And if 2020 has taught us anything, it should be to expect the
    unexpected.

\end{frame}

\section{My Backup Strategies}

\begin{frame}{A bit of history}

    \begin{itemize}[<+->]
        \item For a long time, I just copied my photos and other important
            documents to a hard drive. And I've done this on and off throughout
            the years.
        \item I have a fairly old copy of my photos from about 2012 on DVDs.
        \item I started by using CrashPlan. It allowed P2P backups between
            devices on the same LAN (as well as their servers).
        \item I then migrated to Dropbox as a ``backup'' strategy.
        \item I then migrated from Dropbox to self-hosted Nextcloud. But this
            time, I added a Duplicity backup for my VPS.
        \item Recently, I migrated to Syncthing for file sync and I'm using
            Restic to backup my VPS.
    \end{itemize}

\end{frame}

\begin{frame}{What I backup}

    \begin{itemize}
        \item Documents
        \item Photos
        \item Projects
        \item Dotfiles
        \item System configs
        \item The data for programs that I self-host
    \end{itemize}

\end{frame}

\begin{frame}{Backup method 1: Git}

    I use Git to backup a lot of things. It's really good when all you have is
    text.
    \pause

    \begin{itemize}[<+->]
        \item All of my projects (besides some really old ones) are in Git
            repos. I use either GitHub, GitLab, or Sourcehut to host all of my
            repos.
        \item All of the dotfiles that I care about are stored in a Git repo,
            managed using Chezmoi.
        \item I'm currently in the process of migrating to NixOS on all of my
            machines, and everything is declarative.
    \end{itemize}

\end{frame}

\begin{frame}{Backup method 2: Restic}

    I use Restic to backup everything else. I only backup my Linode VPS, though.

    \pause
    All data gets synced to my VPS, and then from there it gets backed up to
    BackBlaze B2 using Restic.

    \pause
    Restic encrypts the backups by default. It stores data in a structure very
    similar to Git.

\end{frame}

\begin{frame}{Future}

    I want to also figure out a good way to have an on-site non-synchronized
    Restic backup.

\end{frame}

\section{Conclusion}

\begin{frame}{This may be daunting}

    My backup strategy is kinda complicated. Don't let that deter you!

    \pause
    I've built this up over many years and it's almost become an obsession of
    mine. \pause And given that the apocalypse is upon us, maybe it's not that
    bad of an obsession to have.

    \pause
    Start somewhere, even if it's just copying important files to a hard drive,
    and go from there.

\end{frame}

\end{document}
% Local Variables:
% TeX-command-extra-options: "-shell-escape"
% End:
